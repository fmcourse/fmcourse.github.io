\documentclass[hyperref={unicode=true}]{beamer}
\usepackage[utf8]{inputenx}
\usepackage[russian]{babel}

\usepackage{multicol}

\usepackage[pgf]{dot2texi}
\usepackage{tikz}
\usetikzlibrary{shapes, arrows}

\usepackage{listings}
\usepackage{graphicx}

\usepackage{comment}

\title{Лекция 4. Формальная спецификация и верификация Си-программ}
\author{}
\date{}

\usetheme{Warsaw}

\AtBeginSection[] {
	\begin{frame}{Содержание}
		\tableofcontents[currentsection]
	\end{frame}
}
%\overfullrule=5pt

\begin{document}
	\begin{frame}{}
		\titlepage
	\end{frame}

    \begin{frame}{Цель лекции}
    Дать краткое введение в язык формальной спецификации
    Си-программ ACSL и среду формальной верификации AstraVer.
    \end{frame}


    \section{Язык ACSL}


    \begin{frame}[fragile]{ACSL}
    \begin{itemize}
    \item
    ACSL -- ANSI/ISO C Specification Language.
    \item
    Исходный код дополняется \emph{аннотациями}:
    комментариями \verb|/*@ ... */|
    \item
    Запуск верификации: \verb|frama-c -av <file.c>|:
        \begin{enumerate}
        \item
        Генерируются условия верификации на Why3
        \item
        Открывается окошко Why3IDE
        \item
        Пользователь запускает солверы для доказательства условий верификации
        \end{enumerate}
    \end{itemize}
    \end{frame}


    \begin{frame}[fragile]{Аннотации ACSL для спецификации функции}
    Спецификация функции -- это аннотация, которая
    расположена перед заголовком функции.
    \begin{itemize}
    \item
    \verb|requires expr;| -- предусловие
    (можно несколько requires или ни одного)
    \item
    \verb|ensures expr;| -- постусловие
    (можно несколько ensures или ни одного)
    \end{itemize}
    \end{frame}

    \begin{frame}[fragile]{Типы данных}
    \begin{itemize}
    \item
    \verb|integer| -- бесконечный целый тип (только для спецификаций)
    \item
    Все Си-шные типы тоже есть
    \item
    В операциях с \verb|int| и \verb|integer| происходит преобразование
    \verb|int| к \verb|integer|
    \end{itemize}
    \end{frame}

    \begin{frame}[fragile]{Спецификация циклов}
    \begin{itemize}
    \item
    Аннотация цикла записывается перед циклом.
    \item
    Точки сечения -- перед условием цикла.
    \item
    Фундированное множество -- неотрицательные целые числа
    с отношением <<меньше>>.
    \item
    \verb|loop invariant expr;| -- индуктивное утверждение
    (может быть несколько или отсутствовать)
    \item
    \verb|loop variant expr;| -- оценочная функция
    (если ее нет, то она равна 0)
    \end{itemize}
    \end{frame}

    \begin{frame}[fragile]{Предикаты и лоджики}
    Для повышения читаемости спецификаций можно давать имена
    выражениям-предикатам и выражениям-не предикатам:
    \begin{itemize}
    \item
    \verb|predicate positive(integer x) = x >= 0;| --
    определяет имя \verb|positive| как имя предиката для использования в аннотациях
    \item
    \verb|logic integer square(integer x) = x * x;| --
    определяет имя \verb|square| как имя <<функции>> для использования в аннотациях
    \item
    рекурсивные определения поддерживаются, но они могут давать неожиданный результат
    \item
    аннотации используются только для формулирования условий
    верификации и проведения символьных преобразований над ними
    в солверах, так что лоджик -- это не функция (лишь текстуальная замена)
    \end{itemize}
    \end{frame}

    \begin{frame}{Указатели: модель памяти}
    \begin{itemize}
    \item
    Память состоит из бесконечного множества блоков.
    \item    
    Блок имеет фиксированный тип элемента.
    \item
    Блок имеет фиксированный размер (количество элементов).
    \item
    Блок может быть не аллоцирован или аллоцирован.
    \item
    Блоку соответствует бесконечное количество указателей.
    \item
    У любого блока (даже не аллоцированного) есть указатель на начало блока
    (<<базовый адрес>>)
    \item
    Указатели типизированные (из блока).
    \end{itemize}
    \end{frame}

    \begin{frame}[fragile]{Указатели: спецификация}
    \begin{itemize}
    \item
    \verb|\valid(p)|, \verb|\valid(p + (0 .. n-1))| -- валидность
    указателя \verb|p| и диапазона указателей \verb|p|, ..., \verb|p + (n-1)|
    (обе границы включаются в диапазон)
    \item
    \verb|\offset_min(p)|, \verb|\offset_max(p)| -- минимальное
    и максимальное смещение для указателя \verb|p|, чтобы он был валидным,
    если блок указателя аллоцирован;
    \item
    \verb|\base_addr(p) == \base_addr(q)| -- проверка, что \verb|p|
    и указателя находятся в одном блоке
    \item
    \verb|\allocable(p)| -- блок указателя \verb|p| аллоцирован
    \item
    \verb|\freeable(p)| -- блок указателя \verb|p| не аллоцирован и \verb|p| указывает
    на начало блока
    \end{itemize}
    \end{frame}

    \begin{frame}[fragile]{at, old, метки памяти}
    \begin{itemize}
    \item
    \verb|\old(expr)| -- выражение \verb|expr| в момент вызова функции
    (\verb|\old| можно использовать только в постусловии)
    \item
    \verb|\at(expr, L)| -- выражение \verb|expr| в метке памяти \verb|L|
    \item
    Метки памяти -- это Си-метки + \verb|Pre|, \verb|Here|, \verb|Post|
    \item
    Можно указывать метку и так: \verb|\valid{L}(p)|, \verb|\freeable{L}(p)| и т.п.
    \item
    Метки могут быть у \verb|predicate| и \verb|logic|, но написать квантор по меткам нельзя
    (может быть даже несколько меток!)
    \item
    Не путать \verb|\at(*p,L)| и \verb|*\at(p,L)|
    \end{itemize}
    \end{frame}

    \begin{frame}[fragile]{Спецификация фрейма функции}
    \begin{itemize}
    \item
    \verb|assigns Locations;| -- валидная память за пределами \verb|Locations|
    не меняет своего значения при выходе из функции
    \item
    \verb|allocates Locations;| -- валидная память за пределами \verb|Locations|
    не меняет своего статуса аллоцированности при выходе из функции
    (если была аллоцирована, остается такой же; была не аллоцирована,
     остается такой же); \verb|Locations| вычисляется в метке памяти \verb|Post|
    \item
    \verb|frees Locations;| -- то же, что \verb|allocates|, но \verb|Locations|
    вычисляется в метке памяти \verb|Pre|
    \item
    Не путать \verb|allocates| и \verb|allocable|, \verb|frees| и \verb|freeable|
    \end{itemize}
    \end{frame}

%    \begin{frame}[fragile]{Аннотации для верификации}
%    \begin{itemize}
%    \item
%    \verb|assert pred;| -- подсказка, что утверждение \verb|pred| может быть
%    полезно для доказательства
%    \item
%    \verb|ghost|-метки и локальные переменные (тип - только Си-шный),
%    \verb|ghost|-блоки (только с Си-шным кодом): это из-за того, что
%    в Frama-C фронтенд кода принимает только Си-код (не ghost-аннотация
%    анализируется фронтендом отдельно, поэтому с ней нет этих проблем)
%    \item
%    \verb|ghost|-функции
%    \item
%    \verb|lemma| -- леммы
%    \item
%    лемма-функции
%    \end{itemize}
%    \end{frame}

%    \begin{frame}{Behavior}
%    \begin{itemize}
%    \item
%    Можно указать у assert, что он нужен для доказательства
%    определенной аннотации ensures
%    \item
%    То же можно сделать с loop invariant
%    \item
%    Ensures надо поместить в behavior
%    \item
%    Синтаксис смотрите в документации по ACSL
%    \end{itemize}
%    \end{frame}

%    \begin{frame}[fragile]{Лемма-функции}
%    Пример:
%
%    \begin{lstlisting}
%/*@ ghost
%     /@ lemma
%          requires ....
%          ensures ....
%     @/
%     void lemmafunction(....) {
%      ....
%     }
%*/
%    \end{lstlisting}
%    \end{frame}

    \begin{frame}[fragile]{Глобальные инварианты}
    \begin{itemize}
    \item
    Это утверждения, которые выполнены всегда, когда
    программа находится в определенной точке программы
        \begin{itemize}
        \item строгие - везде
        \item слабые - при вызове и при возврате из
        каждой функции
        \end{itemize}
    \item Инвариант глобальных переменных:
    \begin{block}{}
    \verb|global invariant positive: size >= 0;|
    \end{block}
    \item Инвариант (каждой переменной) типа:
    \begin{block}{}
    \begin{verbatim}type invariant valid_array(Array *a) =
a->size >= 0 && \valid(a->data + (0 .. a->size - 1));\end{verbatim}
    \end{block}
    \end{itemize}
    \end{frame}

    \begin{frame}{Проблемы глобальных инвариантов}
    \begin{itemize}
    \item
    Пусть у некоторого типа должен быть инвариант.
    У любой ли функции этот инвариант должен быть
    выполнен хотя бы в слабом смысле?
    \item
    Для функции-конструктора? (а что это в Си?)
    \item
    Для функции-деструктора? (а что это в Си?)
    \item
    Для вспомогательной функции, вызываемой из
    <<публичной>> функции? Любой такой вспомогательной
    функции?
    \end{itemize}
    \end{frame}

    \begin{frame}{Подход AstraVer}
    \begin{itemize}
    \item
    Надо точнее специфицировать, когда должен быть
    выполнен глобальный инвариант. Вводить дополнительные
    конструкции в язык спецификации?
    \item
    В AstraVer глобальные инварианты не
    вставляются автоматически в спецификации функций!
    \item
    Это просто предикаты, которые надо явно
    указать в спецификациях нужных функций.
    \end{itemize}
    \end{frame}

    \begin{frame}{И это не все проблемы}
    \begin{itemize}
    \item
    Пусть некоторый тип является частью другого типа.
    Причем не все корректные значения внутреннего
    типа являются допустимыми в рамках внешнего типа.
    \item
    Каков должен быть инвариант внутреннего типа?
    Как доказать, что модификация переменной
    внутреннего типа не нарушает инварианта
    переменной внешнего типа, если нет доступа
    до переменной внешнего типа?
    \item
    До сих пор нет лучшего ответа на этот вопрос.
    \end{itemize}
    \end{frame}

    \section{Система Frama-C}

    \begin{frame}{Frama-C}
    \begin{itemize}
    \item Система статического анализа Си-программ.
    \item Состоит из ядра и плагинов. Ядро -- это фронтенд
    анализа, плагин -- бекэнд анализа.
    \item Один из плагинов (AstraVer) -- дедуктивная верификация.
    \item Плагины могут взаимодействовать друг с другом
    для компенсации недостатков друг друга. Пример:
    есть плагин, который сам выводит несложные инварианты цикла,
    и уже на этой основе другой плагин сообщает об ошибке.
    Всё это делается полностью автоматически.
    \end{itemize}
    \end{frame}

    \begin{frame}{Работа на уровне исходного кода}
    \begin{itemize}
    \item Исходный код может быть снабжен аннотациями
    для более точного анализа. Для дедуктивной верификации
    аннотациями записывается спецификация, инварианты цикла
    и т.п.
    \item Причем пользователь Frama-C работает
    исключительно на уровне исходного кода (иначе будет тяжело
    автоматизировать комбинирование плагинов).
    \item На уровне исходного кода не доступна его модель на WhyML.
    Эта модель может иметь существенные особенности по сравнению
    с исходным Си-кодом, важные для верификации.
    \end{itemize}
    \end{frame}

    \begin{frame}{Модель кода WhyML не доступна}
    \begin{itemize}
    \item (-) Например, нельзя написать лемму, в которой
    используются функциональные символы из модели памяти.
    \item (+) Плагин \textsl{AstraVer} может самостоятельно
    строить модель программы, применяя различные
    оптимизации для более эффективной верификации.
    \end{itemize}
    \end{frame}

    \begin{frame}{Переносимость Си-программ и верификации}
    \begin{itemize}
    \item
    В Си размеры типов выбирает платформа, а не язык.
    \item
    Frama-C фиксирует размеры типов во фронтенд анализе
    (опциями можно настраивать эти размеры). Дедуктивная
    верификация делается с этими размерами типов.
    \item
    Последовательность вычислений тоже фиксируется
    фронтендом Frama-C.
    \end{itemize}
    \end{frame}

\end{document}
